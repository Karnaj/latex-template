\documentclass[small]{zmdocument}

\usepackage{blindtext}

\title{Ceci est un titre}
\author{Karnaj, pierre\_24, Heziode, Clem}
\licence[./test-images/cc-by-nc-nd_icon.svg]{CC-BY-NC-ND}{https://creativecommons.org/licenses/by-nc-nd/4.0/legalcode}

\logo{./test-images/tuto_logo.png}
\editorLogo[La connaissance pour tous et sans pépins]{./test-images/zestedesavoir.svg}

\smileysPath{./test-smileys}
\makeglossaries

\begin{document}
\maketitle
\tableofcontents
\newpage

\levelOneIntroduction

Ce message s’adresse principalement aux bon connaisseurs de LaTeX : l’équipe de développement a besoin de vous ! \smiley{clin}

\begin{Spoiler}
Voici un secret.

Avec un \externalLink{lien}{https://zestedesavoir.com}!!
\end{Spoiler}

\levelOneTitle{Contexte}

Nous travaillons actuellement sur la refonte de l’outils qui gère la partie \abbr{MD}{Markdown} -> autres formats et nous souhaitons refaire le template actuel LaTeX\textsuperscript{\ref{pandoc}}. Nous utilisons celui de base de Pandoc\textsuperscript{\ref{pandoc}} or cet outil sera remplacé dans le futur. Nous cherchons donc quelques personnes pour nous aider dans la réalisation de macros pour chaque élément spécifique à ZdS.

\footnotetext[1]{\label{pandoc} Parce que Pandoc, c'est un peu l'horreur}

Si vous avez des questions et/ou si vous êtes intéressés, n’hésitez pas à passer sur IRC ou à laisser un message par ici \smiley{diable}

\begin{Spoiler}[Ceci est un spoiler avec titre \textit{custom}, parce que!]
Voici un autre secret.
\end{Spoiler}

\levelTwoTitle{Les \textbf{éléments} \sout{à faire} \textit{faits}}
\levelThreeTitle{Ensemble des éléments classiques}

\begin{itemize}
\item \textbf{Gras}
\item \textit{Italique}
\item \sout{Barré}
\item Indice\textsubscript{X}
\item Exposant\textsuperscript{2}
\item \externalLink{liens}{https://zestedesavoir.com}
\end{itemize}

\begin{Spoiler}
Voici un dernier secret.
\end{Spoiler}

\begin{flushleft}
Texte aligné à gauche
\end{flushleft}

\begin{center}
Texte centré
\end{center}

\begin{flushright}
Texte aligné à droite
\end{flushright}

\begin{enumerate}
\item Listes
\item Liste numérotées
\end{enumerate}

\iframe{https://www.youtube.com/watch?v=dQw4w9WgXcQ}[Vidéo Youtube][Une super vidéo que voilà \smiley{diable}]

\levelTwoTitle{Titre}
\levelThreeTitle{De}
\levelFourTitle{Toute}
\levelFiveTitle{taille}

Et maintenant, une citation:

\begin{Quotation}[Clem]
Zeste de Savoir, la connaissance pour tous et sans pépins \smiley{diable}
\end{Quotation}

La suite, avec des touches \keys{CTRL + A}. Et on peut avoir une ligne avec

\horizontalLine

Voici un peu de code inline: \CodeInline{make test}. Plus ? Plus : \CodeInline{\# ceci est une ligne de code incroyablement longue, comme ça elle passe sur plusieurs lignes}.


Et voici quelques exemples de code python :

\begin{CodeBlock}{python}
def foo(bar):
    if True:
        return 42
    return 0
\end{CodeBlock}

\begin{CodeBlock}[Code légendé]{python}
def foo(bar):
    if True:
        return 42
    return 0
\end{CodeBlock}

\begin{CodeBlock}[Code légendé avec lignes colorés][1-2]{python}
def foo(bar):
    if True:
        return 42
    return 0
\end{CodeBlock}

\begin{CodeBlock}[][1, 3-4]{python}
def foo(bar):
    if True:
        return 42
    return 0
\end{CodeBlock}

\begin{CodeBlock}[][12][10]{python}
def foo(bar):
    if True:
        return 42 # This line is emphasized and is numbered 12.
    return 0
\end{CodeBlock}

Bien sûr, on peut écrire des maths, et on peut même leur mettre une légende.

\[
   1 + 1 = 2
\]
\captionof{equationFloat}{Légende}

Une image hors-texte :

\image{./test-images/logo.png}[Légende de l’image]

On peut aussi avoir une image dans le texte \inlineImage{./test-images/tuto_logo.png}, mais dans ce cas, pas de légende.

\begin{center}
\includegraphics[width=\linewidth]{./test-images/logo.png}
\captionof{figure}{Légende}
\end{center}

Vous saviez que le C est issus de \abbr{AT\&T}{American Telephone and Telegraph Company} ?


Et maintenant, différents blocs:

\begin{Information}
\blindtext
\begin{Question}
\blindtext
\end{Question}
\end{Information}

\begin{Question}
\blindtext
\begin{CodeBlock}[][12][10]{python}
def foo(bar):
    if True:
        return 42 # This line is emphasized and is numbered 12.
    return 0
\end{CodeBlock}
\blindtext
\end{Question}

\begin{Warning}
\blindtext
\end{Warning}

\begin{Error}
\blindtext
\end{Error}

\begin{Information}[Titre d'une information méga importante]
\blindtext
\end{Information}

\begin{Neutral}[Titre de mon bloc neutre]
Un bloc neutre, pour contenir un théorème ou une équation par exemple.
\end{Neutral}

Et finalement, un tableau:

\begin{longtabu}{|c|c|c|} \hline
element & element & element\\ \hline
element & element & element\\ \hline
element & element & element\\ \hline
element & element & element\\ \hline
element & element & element\\ \hline
element & element & element\\ \hline
element & element & element\\ \hline
element & element & element\\ \hline
element & element & element\\ \hline
element & element & element\\ \hline
element & element & element\\ \hline
element & element & element\\ \hline
element & element & element\\ \hline
element & element & element\\ \hline
element & element & element\\ \hline
element & element & element\\ \hline
element & element & element\\ \hline
element & element & element\\ \hline
element & element & element\\ \hline
element & element & element\\ \hline
element & element & element\\ \hline
element & element & element\\ \hline
element & element & element\\ \hline
element & element & element\\ \hline
element & element & element\\ \hline
element & element & element\\ \hline
\caption{Légende}
\end{longtabu}

\levelOneConclusion

\begin{Spoiler}
Et un secret ici pour finir en beauté, avec du code dans le dedans !

\begin{CodeBlock}[][1]{python}
def test(a):
    print(a)
\end{CodeBlock}

Et une image:

\begin{center}
\includegraphics[width=\linewidth]{./test-images/logo.png}
\captionof{figure}{Légende}
\end{center}

Et un tableau:

\begin{longtabu}{|c|c|c|} \hline
element & element & element\\ \hline
element & element & element\\ \hline
element & element & element\\ \hline
element & element & element\\ \hline
element & element & element\\ \hline
\caption{Légende du tableau}
\end{longtabu}

Et une citation:

\begin{Quotation}[me]
\textit{What a pain in the ass!}
\end{Quotation}

Et une équation, tient

\[
\begin{split}
A & = \frac{\pi r^2}{2} \\
 & = \frac{1}{2} \pi r^2
\end{split}
\]

\end{Spoiler}

Conclusion

\levelTwoTitle{
Un titre avec des trucs dangereux: 
C\textsubscript{6}H\textsubscript{6}, \textsuperscript{13}C ,
\smiley{diable}, 
\externalLink{un lien}{https://zestedesavoir.com},
\keys{CTRL + A},
\CodeInline{du code},
\abbr{CQFD}{Ce qu'il fallait démontrer},
\inlineImage{./test-smileys/clin.png}
}

\end{document}
