\documentclass[small]{zmdocument}

\usepackage{blindtext}

\title{On teste les images}
\author{Karnaj, pierre\_24, Heziode, Clem}
\licence[./test-images/cc-by-nc-nd_icon.svg]{CC-BY-NC-ND}{https://creativecommons.org/licenses/by-nc-nd/4.0/legalcode}

\logo{./test-images/tuto_logo.png}
\editorLogo[La connaissance pour tous et sans pépins]{./test-images/zestedesavoir.svg}

\smileysPath{./test-smileys}
\makeglossaries

\begin{document}
\maketitle
\tableofcontents

\begin{LevelOneIntroduction}
\end{LevelOneIntroduction}


Et une image en GIF, ça passe ?

\begin{center}
\includegraphics[width=4cm]{test-images/image_in_gif.gif}
\captionof{figure}{Oui !}
\end{center}

Et une image en SVG, ça passe ?

\begin{center}
\includegraphics[width=6cm]{test-images/image_in_svg.svg}
\captionof{figure}{Oui !}
\end{center}

Et une image avec plus d'un point ?

\begin{center}
\includegraphics[width=5cm]{test-images/image_that_was_in.png.jpg}
\captionof{figure}{Il suffisait de demander !}
\end{center}


\end{document}
